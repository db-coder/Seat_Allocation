\begin{frame}
\titlepage
\end{frame}

\begin{frame}
\frametitle{\centerline{CS 251 Project: A Report by Group 22}}
\begin{block}{Contents}
1. Lab 10: Java Implementation \newline \pause 
2. Lab 11: Python and Django ~\cite{django} \newline \pause
3. Conclusions \newline \pause
4. References
\end{block}
\end{frame}

\begin{frame}
\frametitle{\centerline{Lab 10: java Implementation}}
\begin{block}{Contents}
1. Gale Shapley Allocation ~\cite{Allocation} \newline \pause 
2. Merit List Allocation ~\cite{Allocation} \newline \pause 
3. Common Info
\end{block}
\end{frame}

\begin{frame}
\frametitle{\centerline{1. Gale Shapley Allocation ~\cite{Allocation}}}
\begin{block}{Contents}
(i)   It is relatively simple to implement. \newline \pause 
(ii)  MeritList class isn't needed. \newline \pause 
(iii) We stored the ranks in candidate class and used them. \newline \pause 
(iv)  Foreign candidates can be handled easily after everything is done. 
\end{block}
\end{frame}

\begin{frame}
\frametitle{\centerline{2. Merit List Allocation ~\cite{Allocation}}}
\begin{block}{Contents}
(i)  Checking it's correctness is simply too tough. \newline \pause
(ii) You need to manually evaluate the test cases to test the correctness of algorithm, which is a very tedius task in this case.
\end{block}
\end{frame}

\begin{frame}
\frametitle{\centerline{3. Info}}
\begin{block}{Contents}
(i)   Candidates will map candidate id to the object candidate. \newline \pause 
(ii)  Programs will map the program id to the corresponding object VirtualProgramme. \newline \pause 
(iii) In GS, we stored the categories, \newline
	  1 : General,   2 : OBC,   3 : SC,   4 : ST,\newline 5 : GEPD,   6 : OBCPD,   7 : SCPD,   8 : STPD. \newline \pause 
(iv)  We stored these categories in the Candidate and VirtualProgramme as well. It helped to reduce the code a lot. \newline \pause 
(v)   And then to compare based on two ranks, we used multiplying factor of $10^{5}$, assuming all ranks will be under $10^{5}$.
\end{block}
\end{frame}


\begin{frame}
\frametitle{\centerline{Lab 11: Python and Django ~\cite{django}}}
\begin{block}{Contents}
1. pdf to csv conversion \newline \pause
2. Populating the database \newline \pause
3. Web framework to display a list of elegible courses 
\end{block}
\end{frame}

\begin{frame}
\frametitle{\centerline{1. pdf to csv conversion}}
\begin{block}{Contents}
(i)   Tried using pdfMiner, pdfTables, pdfpy etc, but all to no avail. Finally used pdftotext by import os to mine it. \newline \pause
(ii)  update.py creates an organised csv file output.csv. \newline \pause
(iii) Sorted the courses of output.csv on the basis of their codes. \newline \pause
(iv)  Merged the two csv files which in turn was to be used for populating the sqlite database of our application.
\end{block}
\end{frame}

\begin{frame}
\frametitle{\centerline{2. Populating the database}}
\begin{block}{Contents}
(i)  Populated the sqlite database, using python shell in the django ~\cite{django} environment by \"python3 manage.py shell\". \newline \pause 
(ii) Ran the commands in file populate\_db in the python shell (manually).
\end{block}
\end{frame}

\begin{frame}
\frametitle{\centerline{3. Web framework to display a list of elegible courses}}
\begin{block}{Contents}
(i)   Made models ~\cite{models} for courses and UserProfile class which contained the attributes of the classes (name, id etc) along with their types (models.py \cite{models} ). \newline \pause
(ii)  Added our django ~\cite{django} app jeeinterface in the mysite/urls.py file to create urls such as jeeinterface/login. \newline \pause 
(iii) Implemented the user register-login system using the built-in django-forms ~\cite{django} .\newline \pause 
(iv)  The rank of the logged user is used to find the courses one is elegible for. \newline \pause
(v)   Courses that the user is elegible for, are displayed via college-wise collapsable lists.
\end{block}
\end{frame}

\begin{frame}
\frametitle{\centerline{Conclusions}}
This is our submission for the CS251 project. Hope you enjoyed going through the project report as much as we enjoyed making it.
Any other reviews and/or suggetions are most welcome. Looking forward to hearing from you.
\end{frame}

\begin{frame}
\frametitle{\centerline{References}}
\bibliography{beamer}
\bibliographystyle{ieeetr}
\end{frame}
